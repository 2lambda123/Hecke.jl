\documentclass[a4paper,10pt]{article}
\usepackage{amsfonts}
\usepackage{amsmath}
%\usepackage{eucal}
\usepackage{amscd}
\usepackage{url}
\usepackage{hyperref}
\usepackage{color}
\usepackage{textcomp}
\usepackage{listings}
\urlstyle{sf}

\addtolength{\oddsidemargin}{-0.75in}
\addtolength{\evensidemargin}{-0.75in}
\addtolength{\textwidth}{1.5in}

\newcommand{\Z}{\mathbf{Z}}
\newcommand{\N}{\mathbf{N}}
\newcommand{\HH}{\mathcal{H}}
\newcommand{\Q}{\mathbf{Q}}
\newcommand{\I}{\mathbb{I}}
\newcommand{\C}{\mathbf{C}}
\newcommand{\R}{\mathbf{R}}
\newcommand{\F}{\mathbf{F}}
\newcommand{\Pee}{\mathbb{P}}
\newcommand{\EuO}{\mathcal{O}}
\newcommand{\Qbar}{\overline{\mathbf{Q}}}
\newcommand{\fn}{\hfill[Function]}
\newcommand{\macro}{\hfill[Macro]}
\newcommand{\gmp}{\hfill[GMP]}
\newcommand{\code}{\lstinline}
\newcommand{\desc}[1]{\vspace{-3mm}\begin{quote}#1\end{quote}}
\newcommand{\Mat}{\operatorname{Mat}}

\newcommand{\ljk}[2]{\left(\frac{#1}{#2}\right)}
\newcommand{\modulo}[1]{\;\left(\mbox{mod}\;#1\right)}
\newcommand{\fr}{\mathfrak}

\def\notdivides{\mathrel{\kern-3pt\not\!\kern4.5pt\bigm|}}
\def\nmid{\notdivides}
\def\nsubseteq{\mathrel{\kern-3pt\not\!\kern2.5pt\subseteq}}

\newsavebox{\itembox}
\newcommand*{\descitem}{%
\everyhbox{\everyhbox{}\aftergroup\finishdescitem}%
\setbox\itembox\hbox
}
\newcommand*{\finishdescitem}{%
\everyhbox{}%
\item[\box\itembox]%
}

\lstdefinelanguage{julia}
{
  keywordsprefix=\@,
  morekeywords={
    exit,whos,edit,load,is,isa,isequal,typeof,tuple,ntuple,uid,hash,finalizer,convert,promote,
    subtype,typemin,typemax,realmin,realmax,sizeof,eps,promote_type,method_exists,applicable,
    invoke,dlopen,dlsym,system,error,throw,assert,new,Inf,Nan,pi,im,begin,while,for,in,return,
    break,continue,macro,quote,let,if,elseif,else,try,catch,end,bitstype,ccall,do,using,module,
    import,export,importall,baremodule,immutable,local,global,const,Bool,Int,Int8,Int16,Int32,
    Int64,Uint,Uint8,Uint16,Uint32,Uint64,Float32,Float64,Complex64,Complex128,Any,Nothing,None,
    function,type,typealias,abstract
  },
  sensitive=true,
  morecomment=[l]{\#},
  morestring=[b]',
  morestring=[b]" 
}

\parindent=0pt
\parskip 4pt plus 2pt minus 2pt

%\email{goodwillhart@googlemail.com}

\begin{document}
\lstset{
  language=julia,
  keywordstyle=\bfseries\ttfamily\color[rgb]{0,0.6,0.4},
  identifierstyle=\ttfamily\color[rgb]{0,0.2,0.8},
  commentstyle=\color[rgb]{0.8,0.05,1},
  stringstyle=\color[rgb]{0.8,0.1,0.1},
  basicstyle=\ttfamily,
  showstringspaces=false,
}
%\maketitle
\tableofcontents

\section{Mathematical background and conventions}

\paragraph{Representation matrix of a number field element.}

Let $K$ be an algebraic number field with $\Q$-basis $(\alpha_1,\dotsc,\alpha_d)$.
For an element $\alpha$ of $K$ we call the matrix $M_\alpha \in \Mat_{d \times d} (\Q)$ with
\begin{align*} \alpha \cdot \begin{pmatrix} \alpha_1 \\ \alpha_2 \\ \vdots \\ \alpha_d \end{pmatrix} = \begin{pmatrix} \alpha_1 \\ \alpha_2 \\ \vdots \\ \alpha_d \end{pmatrix} \cdot \alpha = M_\alpha \cdot \begin{pmatrix} \alpha_1 \\ \alpha_2 \\ \vdots \\ \alpha_d \end{pmatrix}
\end{align*}
the \textit{representation matrix} of $\alpha$ (with respect to $(\alpha_1,\dotsc,\alpha_d)$).
In particular, if $\beta$ is another element of $K$ with coefficient vector $(b_1,\dotsc,b_d)$, that is, $\beta = \sum_{i=1}^d b_i \alpha_i$, then we have
\begin{align*} \alpha \cdot \beta = \beta \cdot \alpha = \begin{pmatrix} b_1 & b_2 & \dotsc & b_d \end{pmatrix} \cdot \begin{pmatrix} \alpha_1 \\ \alpha_2 \\ \vdots \\ \alpha_d \end{pmatrix} \alpha = (b_1 \, b_2 \, \dotsc \, b_d ) \cdot M_\alpha \begin{pmatrix} \alpha_1 \\ \alpha_2 \\ \vdots \\ \alpha_d \end{pmatrix}. \end{align*}
Thus, the coefficient vector of $\alpha \cdot \beta$ is just $(b_1\, b_2 \, \dotsc \, b_d ) \cdot M_\alpha$.

\paragraph{Representation matrix of an order element}

Let $\mathcal O$ be an order in an lgebraic number field with $\Z$-basis $(\omega_1,\dotsc,\omega_d)$.
For an element $\alpha$ of $K$ we call the matrix $M_\alpha \in \Mat_{d \times d} (\Z)$ with
\begin{align*} \alpha \cdot \begin{pmatrix} \omega_1 \\ \omega_2 \\ \vdots \\ \omega_d \end{pmatrix} = \begin{pmatrix} \omega_1 \\ \omega_2 \\ \vdots \\ \omega_d \end{pmatrix} \cdot \alpha = M_\alpha \cdot \begin{pmatrix} \omega_1 \\ \omega_2 \\ \vdots \\ \omega_d \end{pmatrix}
\end{align*}
the \textit{representation matrix} of $\alpha$ (with respect to $(\omega_1,\dotsc,\omega_d)$).
In particular, if $\beta$ is another element of $\mathcal O$ with coefficient vector $(b_1,\dotsc,b_d)$, that is, $\beta = \sum_{i=1}^d b_i \omega_i$, then we have
\begin{align*} \alpha \cdot \beta = \beta \cdot \alpha = \begin{pmatrix} b_1 & b_2 & \dotsc & b_d \end{pmatrix} \cdot \begin{pmatrix} \omega_1 \\ \omega_2 \\ \vdots \\ \omega_d \end{pmatrix} \alpha = (b_1 \, b_2 \, \dotsc \, b_d ) \cdot M_\alpha \begin{pmatrix} \omega_1 \\ \omega_2 \\ \vdots \\ \omega_d \end{pmatrix}. \end{align*}
Thus, the coefficient vector of $\alpha \cdot \beta$ is just $(b_1\, b_2 \, \dotsc \, b_d ) \cdot M_\omega$.

\paragraph{Basis matrix of an order.}

Let $K$ be an algebraic number field with $\Q$-basis $(\alpha_1,\dotsc,\alpha_d)$ and $\mathcal O$ a $\Z$-order in $K$ with $\Z$-basis $\Omega = (\omega_1,\dotsc,\omega_d)$.
The \textit{basis matrix} of $\mathcal O$ (with respect to $(\alpha_1,\dotsc,\alpha_d)$ and $\Omega$) is the matrix $M \in \Mat_{d \times d}(\Q)$ with
\begin{align*}
  \begin{pmatrix} \omega_1 \\ \omega_2 \\ \vdots \\ \omega_d \end{pmatrix} =
  M \cdot \begin{pmatrix} \alpha_1 \\ \alpha_2 \\ \vdots \\ \alpha_d \end{pmatrix}
\end{align*}

\paragraph{Basis matrix of an ideal of an order.}

Let $\mathcal O$ be an order of an algebraic number field with $\Z$-basis $\Omega = (\omega_1,\dotsc,\omega_d)$.

\paragraph{Basis matrix of a fractional ideal of an order.}

\section{NfOrder : Orders in absolute number fields}

Orders in number fields are modeled using the type \code{NfOrder}, which has the following definition:

\subsection{Definition}

\begin{lstlisting}
type NfOrder <: Ring             
  nf::NfNumberField              
  basis # Array{NfOrderElem, 1}  
  _basis::Array{nf_elem, 1}      
  basis_mat::FakeFmpqMat         
  basis_mat_inv::FakeFmpqMat     
  discriminant::fmpz             
  isequationorder::Bool          
  index::fmpz                
  parent::NfOrderSet 
end
\end{lstlisting}

Let $\mathcal O$ be an order represented by an object \code{O} of type \code{NfOrder}. 

\begin{description}
\descitem{\code{O.nf}:} Object \code{K} of type \code{NfNumberField}, which itself represents a number field $K$ such that $\mathcal O$ is a $\Z$-order of $K$.

\descitem{\code{O.basis}:} Array of objects of type \code{NfOderElem} which represents elements $\omega_1,\dotsc,\omega_d \in \mathcal O$ such that $(\omega_1,\dotsc,\omega_d)$ is a $\Z$-basis of $\mathcal O$.
(We cannot declare it to be of type \lstinline!Array{NfOrderElem, 1}!, since the definition of the type \code{NfOrder} preceeds the definition of and uses the type \code{NfOrder}).

\descitem{\code{O._basis}:} Array of objects of type \code{nf_elem} with parent \code{O.nf} which represents elements $\omega_1,\dotsc,\omega_d \in K$ such that $(\omega_1,\dotsc,\omega_d)$ is a $\Z$-basis of $\mathcal O$. Note that the objects represented by \code{O.basis} and \code{O._basis} are equal.

\descitem{\code{O.basis_mat}:} Let $\alpha_1,\dotsc,\alpha_d$ be the elements represented by \code{basis(K)} and $M \in \Mat_{d\times d}(\Q)$ the matrix represented by \code{O.basis_mat}. Then $M$ is the basis matrix of $\mathcal O$ (with respect to $(\omega_1,\dotsc,\omega_d)$ and $(\alpha_1,\dotsc,\alpha_d)$).

\descitem{\code{O.basis_mat_inv}:} Let $N \in \Mat_{d \times d}(\Q)$ be the matrix represented by \code{O.basis_mat_inv}. Then $N$ is the inverse of $M$, that is, $M N = N M = \mathbf{1}_d$.

\descitem{\code{O.discriminant}:} Represents the discriminant of $\mathcal O$.

\descitem{\code{O.isequationorder}:} If $f \in \Q[X]$ is the polynomial represented by \code{O.nf.pol}, then \code{O.isequationorder} is set to \code{true}, if and only if $\mathcal O$ is the equation order of $f$.

\descitem{\code{O.index}:} I don't know (yet).

\descitem{\code{O.parent}:} The mandatory parent object. All objects of type \code{NfOrder} representing an order in a fixed number field have this parent object in common.
\end{description}

Note that you should never ever access the field \code{x} of an object of type \code{NfOrder} using the function \code{O.x}. Always use \code{x(O)}!
For example, if you want to access the discriminant, use \code{discriminant(O)}.
(Not all fields are set upon creation).

\subsection{Creation}

Objects of type \code{NfOrder} can be created as follows:

\begin{lstlisting}
  EquationOrder(K::NfNumberField)
\end{lstlisting}

\desc{Create the equation order of the number field represented by \code{K} with defining polynomial \code{K.pol}.}

\begin{lstlisting}
  Order(arr::Array{nf_elem, 1})
\end{lstlisting}
  
\desc{Given a list of number field elements $\alpha_1,\dotsc, \alpha_d \in K$, create the order $\mathcal O$ with $\Z$-basis $(\alpha_1,\dotsc,\alpha_d)$ (It is not checked if this really is really defines a $\Z$-order of $K$.)}

\begin{lstlisting}
  Order(K::NfNumberField, A::FakeFmpqMat)
\end{lstlisting}

\desc{Let $K$ be the number field represented by \code{K} and $\alpha_1,\dotsc,\alpha_d$ the basis represented by \code{basis(K)}.
Let $A \in \Mat_{d \times d}(\Q)$ be the matrix represented by \code{A}. Then this function creates the order of $K$ with basis matrix $A$. It is not checked if this really defines a $\Z$-order of $K$.}

Here are some examples:

\begin{verbatim}
julia> Qx,x = PolynomialRing(QQ, "x");

julia> f = x^12 - x^11 + 5*x^4 + 25
x^12 - 1*x^11 + 5*x^4 + 25

julia> O = EquationOrder(K)
Order of Number field over Rational Field with defining polynomial x^12 - 1*x^11 + 5*x^4 + 25
with Z-basis [1,a,a^2,a^3,a^4,a^5,a^6,a^7,a^8,a^9,a^10,a^11]

julia> OO = Order([K(2), 2*a, 2*a^2, 2*a^3, 2*a^4, 2*a^5, 2*a^6, 2*a^7, 2*a^8, 2*a^9, 2*a^10, 2*a^11])
Order of Number field over Rational Field with defining polynomial x^12 - 1*x^11 + 5*x^4 + 25
with Z-basis [2,2*a,2*a^2,2*a^3,2*a^4,2*a^5,2*a^6,2*a^7,2*a^8,2*a^9,2*a^10,2*a^11]

julia> basis_mat(OO)
FakeFmpqMat with numerator
[2 0 0 0 0 0 0 0 0 0 0 0]
[0 2 0 0 0 0 0 0 0 0 0 0]
[0 0 2 0 0 0 0 0 0 0 0 0]
[0 0 0 2 0 0 0 0 0 0 0 0]
[0 0 0 0 2 0 0 0 0 0 0 0]
[0 0 0 0 0 2 0 0 0 0 0 0]
[0 0 0 0 0 0 2 0 0 0 0 0]
[0 0 0 0 0 0 0 2 0 0 0 0]
[0 0 0 0 0 0 0 0 2 0 0 0]
[0 0 0 0 0 0 0 0 0 2 0 0]
[0 0 0 0 0 0 0 0 0 0 2 0]
[0 0 0 0 0 0 0 0 0 0 0 2]
and denominator 1

julia> basis_mat_inv(OO)
FakeFmpqMat with numerator
[1 0 0 0 0 0 0 0 0 0 0 0]
[0 1 0 0 0 0 0 0 0 0 0 0]
[0 0 1 0 0 0 0 0 0 0 0 0]
[0 0 0 1 0 0 0 0 0 0 0 0]
[0 0 0 0 1 0 0 0 0 0 0 0]
[0 0 0 0 0 1 0 0 0 0 0 0]
[0 0 0 0 0 0 1 0 0 0 0 0]
[0 0 0 0 0 0 0 1 0 0 0 0]
[0 0 0 0 0 0 0 0 1 0 0 0]
[0 0 0 0 0 0 0 0 0 1 0 0]
[0 0 0 0 0 0 0 0 0 0 1 0]
[0 0 0 0 0 0 0 0 0 0 0 1]
and denominator 2

julia> discriminant(O)
26849018109922162628173828125

julia> discriminant(OO)
450451776218075865600000000000000000

julia> discriminant(OO)//discriminant(O)
16777216

julia> 16777216 == (2^12)^2
true
\end{verbatim}

\end{document}
